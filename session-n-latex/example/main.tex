\documentclass{article}

\usepackage{amsmath}

\begin{document}
  \tableofcontents

  \section{Introduction}
  \subsection{Donald Knuth "Fun Fact"}
  Hello World!
  This is the 2nd line.
  This is the 3rd line.

  This is the 2nd paragraph. One time Knuth decided to crosscompile his
  algorithms from binary to quaternary. In the process, he invented DNA.
  Knuth's brain is the most powerful source of energy known to man.
  Unfortunately, harnessing it would destroy mankind. Knuth recently created
  a new sorting algorithm. Its performance is O(1). Knuth can store all
  integers between 0 and 127 in 4 bits. Knuth invented binary.

  \section{More \LaTeX}
  \subsection{How to Make Sections}
  We use the \textbackslash section and \textbackslash subsection command!


  \subsection{Math}
  This is the Pythagorean theorem: $a^2+ b^2 = c^2$. It is very cool.
  \\

  A system is linear if $S\{a_1x_1(t)+a_2x_2(t)\} = a_1S\{x_1\}+a_2f\{x_2\}$.
  \\

  Euler's formula is a mathematical formula in complex analysis that
  establishes the fundamental relationship between the trigonometric
  functions and the complex exponential function. Euler's formula states that
  for any real number $x$: 
  \[e^{i\pi} = cos(x) + i\sin(x)\]
  \\

  The Taylor series for the exponential function ex at a = 0 is

  \[
      \sum_{n=0}^\infty \frac{x^n}{n!} 
      = \frac{x^0}{0!}+\frac{x^1}{1!} + \frac{x^2}{2!} + \frac{x^3}{3!} 
      + \cdots
  \]

  \newpage
  We define a convolution between two function in a continuous domain as the
  integral of the product of the two functions after one is reversed and
  shifted.
  \[
    (f*g)(t) = \int_{-\infty}^\infty f(t-\tau)g(\tau)\, d\tau
  \]
  How about some linear algebra? The set of all 2 by 2 rotation matrices is 
  defined as such:
  \[
    \bigg\{ 
    \begin{pmatrix}
        \cos\theta & -\sin\theta \\
        \sin\theta & \cos\theta
    \end{pmatrix}: 0 \leq \theta < 2\pi
    \bigg\}
  \]

  \subsection{Maxwell's Equations}

  \[
      \nabla\cdot\mathbf{E} = \frac{\rho}{\epsilon_0}
  \]
  \[
      \nabla\cdot\mathbf{B} = 0
  \]
  \[
      \nabla\times\mathbf{E} = -\frac{\partial\mathbf{B}}{\partial t}
  \]
  \[
      \nabla\times\mathbf{B} 
      = -\mu_0\bigg(\mathbf{J}
      +\epsilon_0\frac{\partial\mathbf{E}}{\partial t}\bigg)
  \]

\subsection*{Derivative of Cosine}

\begin{align*}
  \frac{d}{d\theta}\cos\theta
  &=\frac{d}{d\theta}\frac12(e^{j\theta} + e^{-j\theta})\\
  &=\frac12(je^{j\theta}-je^{-j\theta})\\
  &=\frac{j}2(e^{j\theta}-e^{-j\theta})\\
  &=-\frac1{2j}(e^{j\theta}-e^{-j\theta})\\
  &=-\sin\theta
\end{align*}

\end{document}